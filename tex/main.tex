\documentclass[12pt]{report}

\usepackage[utf8]{inputenc}
\usepackage{graphicx}
\usepackage[russian]{babel}
\usepackage{array}

\begin{document}
	%tittlepage
	\title{
		{Matrix tridiagonalization using Given's rotations}\\
		{\large The faculty of Computational Mathematics and Cybernetics of Lomonosov Moscow State University}\\
		%{\includegraphics{university.jpg}}
	}
	\author{Mass Ilya}
	\date{03.02.2022}
	
	\maketitle
	
	\chapter{Naive algorithm}
	
	
	\section{Speed results}
	\begin{table}[h!]
		\centering
		\begin{tabular}{ | c | c | } 
			\hline
			size of matrix & time, ms\\  [0.5ex] 
			\hline\hline
			256 & 54 \\ 
			\hline
			512 & 404 \\ 
			\hline
			1024 & 4313 \\
			\hline
			2048 & 59080 \\
			\hline
		\end{tabular}
		\caption{Results of non-optimized algorithm}
	\end{table}  

	\chapter{Cache Optimization}
	Функция, отвечающая за зануление одной строчки и столбца была обновлена. Теперь она вначале применяет вращения слева
	(меняя строки, зануляя столбец), запоминая углы поворота. Далее, Применяются все повороты для зануления строки построчно,
	Что позволяет обходить матрицу только по строкам.

	\section{Speed results}
	\begin{table}[h!]
		\centering
		\begin{tabular}{ | c | c | }
			\hline
			size of matrix & time, ms\\  [0.5ex]
			\hline\hline
			256 & 28 \\
			\hline
			512 & 229 \\
			\hline
			1024 & 1861 \\
			\hline
			2048 & 15393 \\
			\hline
			4096 & 124157 \\
			\hline
		\end{tabular}
		\caption{Results of cache-optimized algorithm}
	\end{table}

	\chapter{fortran implementation}
	Код переписан на фортран, из-за хранения массивов в памяти по столбцам изменен порядок вращений: сначала столбцы, потом
	по собранным углам вращаются строки

	\section{Speed results}
	\begin{table}[h!]
		\centering
		\begin{tabular}{ | c | c | }
			\hline
			size of matrix & time, ms\\  [0.5ex]
			\hline\hline
			256 & 14 \\
			\hline
			512 & 114 \\
			\hline
			1024 & 961 \\
			\hline
			2048 & 8165 \\
			\hline
			4096 & 67257 \\
			\hline
		\end{tabular}
		\caption{Results of algorithm with fortran implementation}
	\end{table}

	\chapter{Blas usage for C}
	Вращение строк было заменено на метод drot из cblas библиотеки (сохранение углов для поворта для столбцов делается вручную)

	\section{Speed results}
	\begin{table}[h!]
		\centering
		\begin{tabular}{ | c | c | }
			\hline
			size of matrix & time, ms\\  [0.5ex]
			\hline\hline
			256 & 14 \\
			\hline
			512 & 124 \\
			\hline
			1024 & 997 \\
			\hline
			2048 & 8980 \\
			\hline
			4096 & 74376 \\
			\hline
		\end{tabular}
		\caption{Results of algorithm with BLAS C usage}
	\end{table}

	\chapter{Blas usage for Fortran}
	Вращение строк было заменено на метод drot из cblas библиотеки (сохранение углов для поворта для столбцов делается вручную)

	\section{Speed results}
	\begin{table}[h!]
		\centering
		\begin{tabular}{ | c | c | }
			\hline
			size of matrix & time, ms\\  [0.5ex]
			\hline\hline
			256 & 15 \\
			\hline
			512 & 120 \\
			\hline
			1024 & 984 \\
			\hline
			2048 & 8793 \\
			\hline
			4096 & 73021 \\
			\hline
		\end{tabular}
		\caption{Results of algorithm with BLAS Fortran usage}
	\end{table}

	\chapter{lapack usage for C}
	Вызывалась функция dsytd2
	\section{Speed results}
	\begin{table}[h!]
		\centering
		\begin{tabular}{ | c | c | }
			\hline
			size of matrix & time, ms\\  [0.5ex]
			\hline\hline
			256 & 3 \\
			\hline
			512 & 26 \\
			\hline
			1024 & 210 \\
			\hline
			2048 & 2213 \\
			\hline
			4096 & 18461 \\
			\hline
		\end{tabular}
		\caption{Results of algorithm with lapack C usage}
	\end{table}

	\chapter{lapack usage for Fortan}
	Вызывалась функция dsytd2
	\section{Speed results}
	\begin{table}[h!]
		\centering
		\begin{tabular}{ | c | c | }
			\hline
			size of matrix & time, ms\\  [0.5ex]
			\hline\hline
			256 & 3 \\
			\hline
			512 & 24 \\
			\hline
			1024 & 240 \\
			\hline
			2048 & 2217 \\
			\hline
			4096 & 18500 \\
			\hline
		\end{tabular}
		\caption{Results of algorithm with lapack Fortran usage}
	\end{table}
\end{document}